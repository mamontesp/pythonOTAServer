\documentclass[11pt]{article}
%\documentclass{standalone}

\usepackage{mystyle}

\graphicspath{{figures}{figures/}}
\usepackage{textcomp,amsmath,longtable}
\usepackage{latexsym}
\usepackage[usestackEOL]{stackengine}
\usepackage{longtable}
\usepackage[shortlabels]{enumitem}
\usepackage{tikz}
\usepackage{pgf}
\usepackage{listings}
\usepackage{color}

\usetikzlibrary{arrows,automata}

\title{Servidor Firmware Over The Air (FOTA)}
\author{Andrea Montes$^1$}
\date{%
    \today
}

\begin{document}
\maketitle
En el documento refman.pdf encuentran una documentación completa del script que corre el servidor. 

A continuación se presentan las instrucciones para las siguientes tareas en el servidor:
\begin{enumerate}
\item Ejecutar script en localhost para hacer pruebas locales.
\item Modificar archivo del servicio.
\item Levantar el servicio.
\item Conectarse al servidor desde un cliente para descargar un firmware determinado.
\item Conectarse al servidor desde otro computador Linux.
\end{enumerate}

\section{Ejecución local del script del servidor}
%\lstset{frame=tb, language=Python}
\begin{verbatim}
/usr/bin/python /root/fota/otaserver/server.py 
-db "/root/fota/otaserver/database/devices2update.txt" 
-pbf "/root/fota/otaserver/bin" 
-ho "127.0.0.1" -p 4000
\end{verbatim}
Con esta línea en la consola de comandos se pondra disponible el servicio a nivel local. Los argumentos del script se pueden modificar de acuerdo a la ubicación de la base de datos y a la ubicación de los archivos binarios. El script cuenta con un menu de ayuda para identificar los parametros a modificar, como se muestra en seguida.
\begin{verbatim}
usage: server.py [-h] [-db DBNAME] [-pbf PATHBINARYFILES] [-ho HOST] [-p PORT]
                 [-lp LOGPATH]

OTA server in python

optional arguments:
  -h, --help            show this help message and exit
  -db DBNAME, --dbname DBNAME
                        Database file name (path included)
  -pbf PATHBINARYFILES, --pathbinaryfiles PATHBINARYFILES
                        Binary files path
  -ho HOST, --host HOST
                        Host IP
  -p PORT, --port PORT  Port to establish communication
  -lp LOGPATH, --logpath LOGPATH
                        Path to save logging

\end{verbatim}


Para probar la ejecución del servidor, se ejecuta el archivo ubicado /root/fota/otaserver/tests/client.py. En la consola en la que se ejecute el cliente, debe aparecer una salida como sigue:
\begin{verbatim}
First data @4#
Update has been approved
Received data S1130000F817002011040000091200000912000072
Received data S11300100912000009120000091200000912000070
Received data S11300200912000009120000091200000912000060
\end{verbatim}

\section{Modificar el archivo del servicio}
El archivo del servicio está ubicado en /lib/systemd/system/otaserver.service.

\begin{verbatim}
[Unit]
Description=Ota Service
After=multi-user.target
Conflicts=getty@tty1.service

[Service]
Type=simple
ExecStart=/usr/bin/python /root/fota/otaserver/server.py -db "/root/fota/otaserver/database/devices2update.txt" -pbf "/root/fota/otaserver/bin" -ho "" -p 4000
StandardInput=tty-force

[Install]
WantedBy=multi-user.target
\end{verbatim}

Para modificar los parámetros del script como servicio se debe modificar la línea ExecStart.

\section{Levantar el servicio.}
El archivo /root/fota/otaserver/serverActivation contiene las instrucciones para levantar el servicio en caso de que se haya modificado el script o en su defecto se haya caído.

\begin{verbatim}
#//bin/bask
clear
echo "OTA server as service update"

SERVICE=otaserver.service

systemctl stop $SERVICE
systemctl daemon-reload
systemctl enable $SERVICE
systemctl start $SERVICE
systemctl status $SERVICE
\end{verbatim}

\section{Conectarse al servidor desde un cliente para descargar un firmware determinado.}
El cliente debe enviar un mensaje con un paquete(estructura) que contenga:
\begin{itemize}
\item @[1 byte]: Indicador de inicio de trama
\item MCUID [10 bytes]
\item Modelo [1 byte]
\item HW VER [1 byte]
\item FW VER [1 byte]
\item \#[1 byte]: Indicador de finalización de trama
\end{itemize}

Por ejemplo, el formato de la trama debe ser como se muestra a continuación
\begin{itemize}
\item $@000120004915215096595514111415\#$
\item $@000110002359311096595514111415\#$
\end{itemize}   

\section{Conexión al servidor desde otro computador Linux}
\begin{enumerate}
\item Crear un archivo key a partir de un archivo ppk que contiene una clave privada para acceder al servidor fota (el ubicado en Digital Ocean).

\begin{verbatim}
puttygen fota_private_key.ppk -O private-openssh -o fotaserver.key
\end{verbatim}

\item Copiar el archivo .key a la ubicación /home/\emph{username}/.ssh.

\item Crear un archivo config en la ubicación /home/\emph{username}/.ssh (De ya estar creado incuir estas lineas al final)

\begin{verbatim}
Host OTAServer
HostName 209.97.145.137
User root
IdentityFile ~/.ssh/fotaserver.key
\end{verbatim}

\item Acceder al servidor desde la línea de comandos
\begin{verbatim}
ssh OTAServer
\end{verbatim}
Debe aparecer en la consola la siguiente respuesta:
\begin{verbatim}
Welcome to Ubuntu 18.04.2 LTS (GNU/Linux 4.15.0-52-generic x86_64)

 * Documentation:  https://help.ubuntu.com
 * Management:     https://landscape.canonical.com
 * Support:        https://ubuntu.com/advantage

  System information as of Tue Jul 30 03:24:34 UTC 2019

  System load:  0.0                Processes:           85
  Usage of /:   10.0% of 24.06GB   Users logged in:     0
  Memory usage: 18%                IP address for eth0: 209.97.145.137
  Swap usage:   0%                 IP address for eth1: 10.132.183.46

 * Canonical Livepatch is available for installation.
   - Reduce system reboots and improve kernel security. Activate at:
     https://ubuntu.com/livepatch

31 packages can be updated.
0 updates are security updates.


*** System restart required ***
Last login: Mon Jul 29 21:43:57 2019 from 186.28.207.132
root@rdaq-fota:~# 
\end{verbatim}
\end{enumerate}

\end{document}